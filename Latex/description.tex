\documentclass[a4paper, 14pt]{article}
\usepackage[14pt]{extsizes}
\usepackage[T2A]{fontenc}
\usepackage[utf8]{inputenc}
\usepackage[english, russian]{babel}
\usepackage{amsfonts}
\usepackage{amsmath}
\usepackage{enumitem}
\usepackage{geometry}
\usepackage{placeins}
\usepackage{amsmath}
\usepackage{blindtext}
\usepackage{graphicx}
\usepackage{qtree}
\usepackage{indentfirst}
\usepackage{xcolor}
\usepackage{hyperref}
\geometry{left=2cm}
\geometry{right=1.5cm}
\geometry{top=1cm}
\geometry{bottom=2cm}
\pagestyle{empty}
\usepackage{multirow}
\usepackage{tikz}
\usepackage{tkz-graph}
\usetikzlibrary{arrows}
\definecolor{linkcolor}{HTML}{799B03} % цвет ссылок
\definecolor{urlcolor}{HTML}{799B03} % цвет гиперссылок
\hypersetup{pdfstartview=FitH,  linkcolor=linkcolor,urlcolor=urlcolor, colorlinks=true}
\usepackage{amsmath, amsfonts, amssymb, amsthm, mathtools}
\numberwithin{equation}{section}

\pagestyle{plain}
\setcounter{page}{1}
\pagenumbering{arabic}
\begin{document}
\begin{center}
    \textbf{AMC bonus task}\\
    \textbf{Работу выполнили: Петрушина Ксения и Моложавенко Александр, Б05-906 }
\end{center}


Данная криптосистема устроена следующим образом:
	\begin{enumerate}
		\item[i)] Задаётся целочисленное значение $module$, далее случайным образом генерируются ключи: приватный ($A, S, E, ASE$) и публичный ($A, ASE$). у $A$ каждый элемент в диапазоне от $1$ до $module$, у $E, S$ в диапазоне от $-module/2 + 1$ до $module/2$. $ASE$ вычисляется по формуле:
		\begin{equation*}
			A \cdot S + E \mod{module} 	
		\end{equation*} 
		\item[ii)] Сообщение $msg$ подписывается по алгоритму, который сначала генерирует векторы $y_1, y_2$ в том же диапазоне, что и $E, S$, затем вычисляется новый вектор $w$ следующим образом:
		\begin{equation*}
			w = y_1 \cdot A + y_2 \mod{module}
		\end{equation*}

		Затем вычисляется ещё один вектор $c$ по формуле:
		\begin{equation*}
			c = w + hash(msg) \mod{module}
		\end{equation*}

		Далее подсчитываются очень важные векторы $z_1, z_2$:
		\begin{equation*}
			z_1 = S\cdot c + y_1 \mod{module}
		\end{equation*}
		\begin{equation*}
			z_2 = E\cdot c + y_2 \mod{module}
		\end{equation*}

		Кортеж из трёх значений $\{c, z_1, z_2\}$ "--- наша подпись.
		\item[iii)] Проверка подписи составляет вычисление следующего предиката:
		\begin{equation}\label{eq1}
			A\cdot z_1 + z_2 - ASE\cdot c + hash(msg) \equiv c \mod{module}
		\end{equation}

		При этом $z_1, z_2$ не равны нулевому вектору.
	\end{enumerate}

	Наша задача по публичному ключу $\{A, AES\}$ подобрать подходящие $S^*, E^*$ так, чтобы предикат возвращал на них единицу. Вычислить их можно напимер из тождества на $ASE$ ($ASE = A\cdot S + E \mod{module}$). Наши решения отличаются лишь тем, что в решении Александра вектор $S^*$ берётся тождественно равным вектору, каждая координата которого равна $1$, а в решении Ксении $S^*$ генерируется случайно в диапазоне от $1$ до $module$. Затем отличий нет и $E^*$ вычисляется по формуле $E^* = ASE - A\cdot S^* \mod{module}$. 

	Алгоритм очевидно полиномиальный от $module$ (всю сложность составляет вычисление $\mod{module}$, принимая остальные арифметические вычисления за константу).

	Взлом происходит корректно, поскольку при таких $S^*, E^*$ тождество \eqref{eq1} выполняется:
	\begin{equation*}
		A \cdot (S^*\cdot c + y_1) + (E^*\cdot c + y_2) - (A\cdot S^*  + E^*) \cdot c + hash(msg) \equiv
	\end{equation*}
	\begin{equation*}
		\equiv A\cdot y_1 + y_2 + hash(msg) \equiv w + hash(msg) \equiv c 
	\end{equation*}
\end{document}